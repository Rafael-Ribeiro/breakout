\documentclass[a4paper]{article}

\usepackage[portuguese,english]{babel}
\usepackage[utf8]{inputenc}
\usepackage[T1]{fontenc}

\newcommand{\documentTitle}{Software Reuse \\ Final Project}
\newcommand{\pdfTitle}{Software Reuse: Final Project}
\newcommand{\documentAuthors}{João Rafael (2008111876, jprafael@student.dei.uc.pt) \\ José Ribeiro (2008112181, jbaia@student.dei.uc.pt)}

\title{\documentTitle}
\author{\documentAuthors}

\usepackage{hyperref}
\hypersetup{
	pdftitle = \pdfTitle
	,pdfauthor = \documentAuthors
	,pdfsubject = {Software Reuse: Final Project}
	,pdfkeywords = {Software Reuse} {Design Patterns}
	,pdfborder = {0 0 0}
}

\usepackage{subfig}
\usepackage{amsmath}
\usepackage{array}
\usepackage{anysize}
\usepackage{lscape}
\usepackage[pdftex]{graphicx}
\usepackage[table]{xcolor}
\usepackage{caption}

\marginsize{3.5cm}{3.5cm}{3cm}{3cm}

\makeatletter

\begin{document}
\renewcommand{\figurename}{Figure}
\maketitle
\cleardoublepage

\tableofcontents
\cleardoublepage

\setlength{\parindent}{1cm}
\setlength{\parskip}{0.3cm}

\hyphenation{}

% \begin{center}
% 	\includegraphics[width=1.00\textwidth]{images/peak.png}
% 	\captionof{figure}{QRS complex detection.}
% 	\label{fig:peak}
% \end{center}

\section{Introduction}
\section{Architecture Overview}
%% TODO: GUI
% \texttt{BreakoutLauncher}
% \texttt{BreakoutLauncherPlayer}
% \texttt{BreakoutGame}
% \texttt{BreakoutFrame}

%% TODO: Physics
% \texttt{World}
% \texttt{Body}
% \texttt{Movable}
% \texttt{Box}
% \texttt{Circle}
% \texttt{Contact}
% \texttt{AbstractCollisionMediator}

%% TODO: Game
% \texttt{BreakoutWorld}
% \texttt{Drawable}
% \texttt{Paddle} (mention states and PaddleFactory)
% \texttt{Ball} (mention states and BallFactory)
% \texttt{Brick} (mention states and BrickFactory)
% \texttt{Bonus} (mention BonusFactory)
% \texttt{BallBonus}
% \texttt{FireBonus}
% \texttt{GlassBonus}
% \texttt{PhantomBonus}
% \texttt{RadiusBonus}
% \texttt{SpeedBonus}
% \texttt{WidthBonus}

\subsection{Gameplay}
\noindent Por cada \texttt{BreakoutLauncherPlayer}, um \texttt{Player} e um \texttt{Paddle} são criados. A classe \texttt{Player} é abstracta, uma vez que o método \texttt{step} é puramente abstracto; tal acontece para obrigar à sua implementação pelas classes \texttt{HumanPlayer} e \texttt{CPUPlayer}. A classe \texttt{Player} contém 3 métodos que permitem controlar o seu \texttt{Paddle} (\texttt{left}, \texttt{right} e \texttt{stop}), chamados nessa implementação de \texttt{step}. No caso de \texttt{HumanPlayer}, a invocação destas funções é dependente de um método de \textit{input}, o teclado (recorrendo à classe \texttt{Keyboard}); por outro lado, um \texttt{CPUPlayer} vê os seus métodos de controlo invocados por uma \texttt{CPUStrategy} (abstracta). As suas concretizações são \texttt{ClosestBallCPUStrategy} e \texttt{FirstBallCPUStrategy}, estratégias distintas de decisão. O acesso a estas estratégias é feito através de \texttt{CPUStrategyMultiton}, onde as várias estratégias se registam.

\clearpage

\section{Design Patterns}
\subsection{Creational Patterns}
\subsubsection{Prototype}
\noindent Durante o \textit{parsing} do mapa de jogo, a adição de novos \texttt{Brick}s ao \texttt{BreakoutWorld} é implementada segundo o DP Prototype; dadas as características semelhantes de um grande conjunto de \texttt{Brick}s num dado mapa, foi do nosso entender que copiar protótipos dos vários tijolos diferentes seria uma boa abordagem ao problema; tal tarefa torna-se possível pois o ficheiro de mapa possui, antes da descrição do mapa, o conjunto de \texttt{Brick}s e as suas propriedades, pelo que qualquer um deles já se encontrará criado na altura de os instanciar (copiando-os).

\subsubsection{Singleton}
%% TODO:
% the 3 strategies

\subsubsection{Multiton}
%%

% \subsection{Structural Patterns}
% \subsubsection{NONE!}
%% FIXME!

\subsection{Behavioral Patterns}
\subsubsection{Chain of Responsability}
%%

\subsubsection{Mediator}
%%

\subsubsection{Memento}
%% 

\subsubsection{Null Object}
%% TODO: NullCPUStrategy

\subsubsection{State}
%% 

\subsubsection{Strategy}
%% 

\subsubsection{Template Method}
%% 

\end{document}
